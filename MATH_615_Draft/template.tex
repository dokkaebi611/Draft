% Options for packages loaded elsewhere
\PassOptionsToPackage{unicode}{hyperref}
\PassOptionsToPackage{hyphens}{url}
\PassOptionsToPackage{dvipsnames,svgnames,x11names}{xcolor}
%
\documentclass[
  letterpaper,
  DIV=11,
  numbers=noendperiod]{scrartcl}

\usepackage{amsmath,amssymb}
\usepackage{iftex}
\ifPDFTeX
  \usepackage[T1]{fontenc}
  \usepackage[utf8]{inputenc}
  \usepackage{textcomp} % provide euro and other symbols
\else % if luatex or xetex
  \usepackage{unicode-math}
  \defaultfontfeatures{Scale=MatchLowercase}
  \defaultfontfeatures[\rmfamily]{Ligatures=TeX,Scale=1}
\fi
\usepackage{lmodern}
\ifPDFTeX\else  
    % xetex/luatex font selection
\fi
% Use upquote if available, for straight quotes in verbatim environments
\IfFileExists{upquote.sty}{\usepackage{upquote}}{}
\IfFileExists{microtype.sty}{% use microtype if available
  \usepackage[]{microtype}
  \UseMicrotypeSet[protrusion]{basicmath} % disable protrusion for tt fonts
}{}
\makeatletter
\@ifundefined{KOMAClassName}{% if non-KOMA class
  \IfFileExists{parskip.sty}{%
    \usepackage{parskip}
  }{% else
    \setlength{\parindent}{0pt}
    \setlength{\parskip}{6pt plus 2pt minus 1pt}}
}{% if KOMA class
  \KOMAoptions{parskip=half}}
\makeatother
\usepackage{xcolor}
\setlength{\emergencystretch}{3em} % prevent overfull lines
\setcounter{secnumdepth}{-\maxdimen} % remove section numbering
% Make \paragraph and \subparagraph free-standing
\makeatletter
\ifx\paragraph\undefined\else
  \let\oldparagraph\paragraph
  \renewcommand{\paragraph}{
    \@ifstar
      \xxxParagraphStar
      \xxxParagraphNoStar
  }
  \newcommand{\xxxParagraphStar}[1]{\oldparagraph*{#1}\mbox{}}
  \newcommand{\xxxParagraphNoStar}[1]{\oldparagraph{#1}\mbox{}}
\fi
\ifx\subparagraph\undefined\else
  \let\oldsubparagraph\subparagraph
  \renewcommand{\subparagraph}{
    \@ifstar
      \xxxSubParagraphStar
      \xxxSubParagraphNoStar
  }
  \newcommand{\xxxSubParagraphStar}[1]{\oldsubparagraph*{#1}\mbox{}}
  \newcommand{\xxxSubParagraphNoStar}[1]{\oldsubparagraph{#1}\mbox{}}
\fi
\makeatother


\providecommand{\tightlist}{%
  \setlength{\itemsep}{0pt}\setlength{\parskip}{0pt}}\usepackage{longtable,booktabs,array}
\usepackage{calc} % for calculating minipage widths
% Correct order of tables after \paragraph or \subparagraph
\usepackage{etoolbox}
\makeatletter
\patchcmd\longtable{\par}{\if@noskipsec\mbox{}\fi\par}{}{}
\makeatother
% Allow footnotes in longtable head/foot
\IfFileExists{footnotehyper.sty}{\usepackage{footnotehyper}}{\usepackage{footnote}}
\makesavenoteenv{longtable}
\usepackage{graphicx}
\makeatletter
\newsavebox\pandoc@box
\newcommand*\pandocbounded[1]{% scales image to fit in text height/width
  \sbox\pandoc@box{#1}%
  \Gscale@div\@tempa{\textheight}{\dimexpr\ht\pandoc@box+\dp\pandoc@box\relax}%
  \Gscale@div\@tempb{\linewidth}{\wd\pandoc@box}%
  \ifdim\@tempb\p@<\@tempa\p@\let\@tempa\@tempb\fi% select the smaller of both
  \ifdim\@tempa\p@<\p@\scalebox{\@tempa}{\usebox\pandoc@box}%
  \else\usebox{\pandoc@box}%
  \fi%
}
% Set default figure placement to htbp
\def\fps@figure{htbp}
\makeatother
% definitions for citeproc citations
\NewDocumentCommand\citeproctext{}{}
\NewDocumentCommand\citeproc{mm}{%
  \begingroup\def\citeproctext{#2}\cite{#1}\endgroup}
\makeatletter
 % allow citations to break across lines
 \let\@cite@ofmt\@firstofone
 % avoid brackets around text for \cite:
 \def\@biblabel#1{}
 \def\@cite#1#2{{#1\if@tempswa , #2\fi}}
\makeatother
\newlength{\cslhangindent}
\setlength{\cslhangindent}{1.5em}
\newlength{\csllabelwidth}
\setlength{\csllabelwidth}{3em}
\newenvironment{CSLReferences}[2] % #1 hanging-indent, #2 entry-spacing
 {\begin{list}{}{%
  \setlength{\itemindent}{0pt}
  \setlength{\leftmargin}{0pt}
  \setlength{\parsep}{0pt}
  % turn on hanging indent if param 1 is 1
  \ifodd #1
   \setlength{\leftmargin}{\cslhangindent}
   \setlength{\itemindent}{-1\cslhangindent}
  \fi
  % set entry spacing
  \setlength{\itemsep}{#2\baselineskip}}}
 {\end{list}}
\usepackage{calc}
\newcommand{\CSLBlock}[1]{\hfill\break\parbox[t]{\linewidth}{\strut\ignorespaces#1\strut}}
\newcommand{\CSLLeftMargin}[1]{\parbox[t]{\csllabelwidth}{\strut#1\strut}}
\newcommand{\CSLRightInline}[1]{\parbox[t]{\linewidth - \csllabelwidth}{\strut#1\strut}}
\newcommand{\CSLIndent}[1]{\hspace{\cslhangindent}#1}

\KOMAoption{captions}{tableheading}
\makeatletter
\@ifpackageloaded{caption}{}{\usepackage{caption}}
\AtBeginDocument{%
\ifdefined\contentsname
  \renewcommand*\contentsname{Table of contents}
\else
  \newcommand\contentsname{Table of contents}
\fi
\ifdefined\listfigurename
  \renewcommand*\listfigurename{List of Figures}
\else
  \newcommand\listfigurename{List of Figures}
\fi
\ifdefined\listtablename
  \renewcommand*\listtablename{List of Tables}
\else
  \newcommand\listtablename{List of Tables}
\fi
\ifdefined\figurename
  \renewcommand*\figurename{Figure}
\else
  \newcommand\figurename{Figure}
\fi
\ifdefined\tablename
  \renewcommand*\tablename{Table}
\else
  \newcommand\tablename{Table}
\fi
}
\@ifpackageloaded{float}{}{\usepackage{float}}
\floatstyle{ruled}
\@ifundefined{c@chapter}{\newfloat{codelisting}{h}{lop}}{\newfloat{codelisting}{h}{lop}[chapter]}
\floatname{codelisting}{Listing}
\newcommand*\listoflistings{\listof{codelisting}{List of Listings}}
\makeatother
\makeatletter
\makeatother
\makeatletter
\@ifpackageloaded{caption}{}{\usepackage{caption}}
\@ifpackageloaded{subcaption}{}{\usepackage{subcaption}}
\makeatother

\usepackage{bookmark}

\IfFileExists{xurl.sty}{\usepackage{xurl}}{} % add URL line breaks if available
\urlstyle{same} % disable monospaced font for URLs
\hypersetup{
  pdftitle={Journal format to use as template},
  pdfauthor={Sandesh Manjunath Raykar; Daniel Lee},
  pdfkeywords={health effort, income, checkup},
  colorlinks=true,
  linkcolor={blue},
  filecolor={Maroon},
  citecolor={Blue},
  urlcolor={Blue},
  pdfcreator={LaTeX via pandoc}}


\title{Journal format to use as template}
\author{Sandesh Manjunath Raykar \and Daniel Lee}
\date{}

\begin{document}
\maketitle
\begin{abstract}
This study examines two related questions using data from the 2016--2017
American Health Values Survey. The first investigates whether
individuals who report putting greater effort into maintaining their
health are less likely to have been diagnosed by a doctor with high
blood pressure. The second explores the association between annual
household income and the time since an individual's last routine
checkup. Using logistic and ordinal regression analyses, the findings
suggest that higher self-reported health effort is associated with lower
odds of a high blood pressure diagnosis, and that higher income predicts
more recent medical checkups. These results highlight socioeconomic and
behavioral disparities in health outcomes and access to preventive care.
\end{abstract}


\subsection{Introduction and Background}\label{sec-intro}

Preventive healthcare plays a critical role in early disease detection
and reducing long-term health costs. Regular checkups, screenings, and
healthy lifestyle behaviors help identify potential health risks before
they become serious, ultimately improving quality of life and reducing
the financial burden on individuals and the healthcare system. Despite
these benefits, engagement in preventive health practices varies widely
among Americans, reflecting differences in awareness, motivation, and
access to care.

To better understand patterns in Americans' health, this study takes two
complementary perspectives. The first focuses on individual effort to
maintain or improve health, capturing personal behaviors such as
exercise, healthy eating, and proactive health management. The second
examines access to routine medical checkups influenced by income level,
representing the structural and socioeconomic conditions that shape
preventive healthcare utilization. Together, these perspectives provide
a more holistic understanding of how both personal and contextual
factors contribute to preventive health outcomes in the United States.

Previous research shows that family income strongly influences access to
medical care and preventive services. Adults with lower income levels
face greater barriers to care, often forgoing medical visits or
prescriptions due to cost, while even brief insurance disruptions reduce
preventive services such as routine checkups. Building on this evidence,
the present study investigates how socioeconoㅁmic
factors---particularly annual household income---relate to both health
effort and the time since one's last routine checkup. By analyzing these
relationships, this study seeks to highlight socioeconomic disparities
in preventive healthcare among U.S. adults and contribute to a deeper
understanding of the behavioral and structural determinants of health.

\subsection{Study Design and Data collection}\label{sec-design}

This study employs a cross-sectional observational design using data
from the 2016--2017 American Health Values Survey, a nationally
representative dataset of U.S. adults. The analysis focuses on
respondents aged 18 and older who provided complete information on key
variables, including self-reported effort in maintaining the health,
blood pressure diagnosis, annual household income, and time since last
routine medical checkup. Data were collected through structured survey
questionnaires administered online and by phone, ensuring coverage
across diverse demographic and socioeconomic groups. This design allows
for examining associations between behavioral and socioeconomic factors
and health outcomes without manipulating variables or establishing
causality.

This study utilized a cross-sectional observational design drawing upon
data from the 2016--2017 American Health Values Survey (AHVS), conducted
by NORC at the University of Chicago with support from the Robert Wood
Johnson Foundation. The AHVS is a nationally representative survey
designed to assess the health beliefs, values, and behaviors of U.S.
adults across diverse demographic and regional contexts. The analysis in
this study was restricted to respondents aged 18 years and older who
provided complete responses to key variables, including self-reported
health effort, blood pressure diagnosis, annual household income, and
time since last routine medical checkup.

Data were collected using a multimode survey design, incorporating
online (CAWI), mail (self-administered paper questionnaires), and
telephone interviews (CATI). This mixed-method approach enhanced
accessibility for respondents with varying levels of internet access and
literacy, ensuring more comprehensive coverage across urban and rural
populations. The survey was available in both English and Spanish, and
participants received small prepaid and contingent incentives to improve
response rates The AHVS and the related Sentinel Community Surveys
collected detailed information on participants' health values, personal
health priorities, self-efficacy, access to care, and social
determinants of health. The structured questionnaire captured both
quantitative and categorical data, allowing for the examination of
associations between behavioral, social, and economic factors and
various health outcomes without any manipulation of variables. Because
of its cross-sectional nature, this design allows for identifying
correlations and group differences but does not infer causality.

\subsection{Methods}\label{methods}

Use separate sub-sections for data preparation and statistical analysis
methods.

\paragraph{Data Preparation}\label{sec-dataprep-methods}

Data for this study were obtained from the 2016--2017 American Health
Values Survey. The dataset was first cleaned and filtered to include
only respondents aged 18 years and older to focus the analysis on
adults. Observations with missing or incomplete responses for key
analytical variables---specifically health effort, annual household
income, and time since last routine medical checkup---were excluded to
ensure data quality and consistency. This filtering process helped
minimize bias that could result from incomplete information.

Before computing summary measures, several data-cleaning procedures were
implemented. Responses coded as 77 (``Don't know'') or 99 (``No
answer'') were treated as invalid and replaced with missing values (NA)
across relevant variables, particularly the health activity items
(act1--act7) that contributed to the health effort measure. A new
variable, healtheffort, was then created as the mean of act1 through
act7 for each respondent, with missing values excluded from the
calculation. This variable reflects an individual's overall effort
toward maintaining or improving their health.

The income and checkup variables underwent the same cleaning and
recoding procedures. Responses coded as 77 (``Don't know'') and 99
(``Refused'') were treated as missing values (NA). The income variable
was categorized into eight ordered levels, ranging from less than
\$15,000 to \$150,000 or more, while the checkup variable was recoded
into five ordered groups, from within the past year to never. Data
integrity checks were then performed to ensure that all invalid codes
were removed and all transformations were accurately implemented. These
procedures produced a reliable dataset suitable for statistical analysis
and modeling.

\paragraph{Statistical Analysis}\label{sec-analysis-methods}

The statistical analysis for this study was guided by the analytical
framework used in the American Health Values Survey (AHVS) and the
Sentinel Community Health Values Surveys (SCHVS). The AHVS employed
k-means clustering, a widely used unsupervised classification technique,
to identify distinct groups of respondents based on their health values,
beliefs, and behaviors. This method partitions the dataset into k
mutually exclusive clusters by minimizing the within-cluster variance
while maximizing the differences between cluster centroids.

Following the AHVS methodology, several solutions ranging from four to
ten clusters were examined to determine the most meaningful group
segmentation. The final cluster solution was selected based on
statistical indicators such as the cubic clustering criterion and the
Pseudo F statistic, ensuring that the identified typology best captured
variation in respondents' health attitudes and behaviors. Demographic
and descriptive measures were then compared across clusters to confirm
the face validity of the typology, ensuring that the observed groupings
aligned with known differences in health, political, and socioeconomic
characteristics among U.S. adults.

For the present analysis using the cleaned AHVS dataset, descriptive
statistics were first computed to summarize respondent demographics and
key health indicators (e.g., health effort, income level, and time since
last routine checkup). Bivariate tests, including chi-square and
correlation analyses, were used to assess relationships between
categorical and continuous variables. Where applicable, multiple linear
and logistic regression models were applied to examine associations
between behavioral and socioeconomic predictors and health outcomes such
as blood pressure diagnosis, while controlling for demographic
covariates. All data cleaning, transformation, and analyses were
performed using R (version 4.3.2). The tidyverse, dplyr, and ggplot2
packages were utilized for data manipulation and visualization, while
stats functions supported regression modeling and significance testing.
Statistical significance was set at p \textless{} 0.05.

\subsection{Results}\label{sec-results}

\subsubsection{Q \textasciitilde{} C Association: Income vs.~Usual Place
for Care}\label{q-c-association-income-vs.-usual-place-for-care}

We examined how household income (numeric code) varies by usual place
for care (\texttt{checkup\_cat}).\\
Descriptive statistics are presented in \textbf{Table 1}, model
estimates in \textbf{Table 2},\\
and income distribution by care setting is illustrated in \textbf{Figure
1}.

\#\textbar{} label: tbl-qc-desc \#\textbar{} tbl-cap: ``Table 1. Income
summary by usual place for care.'' \#\textbar{} echo: false \#\textbar{}
message: false \#\textbar{} warning: false

library(dplyr) library(knitr)

qc\_desc \textless- clean \textbar\textgreater{} mutate(income\_num =
as.numeric(income)) \textbar\textgreater{} group\_by(checkup\_cat)
\textbar\textgreater{} summarise( n = sum(!is.na(income\_num)), mean =
mean(income\_num, na.rm = TRUE), sd = sd(income\_num, na.rm = TRUE),
median = median(income\_num, na.rm = TRUE), IQR = IQR(income\_num, na.rm
= TRUE), .groups = ``drop'' )

knitr::kable(qc\_desc, digits = c(0,0,2,2,2,2))

\subsection{Discussion, Conclusion}\label{sec-discussino}

Interpret your results, explain their meaning, and place them in context
with existing knowledge.

\begin{itemize}
\tightlist
\item
  Summarize the main findings and discuss their implications or
  applications.
\item
  Address limitations of the study and suggest directions for future
  research.
\end{itemize}

\subsection{References}\label{references}

\phantomsection\label{refs}
\begin{CSLReferences}{0}{1}
\end{CSLReferences}

\subsection{Appendix}\label{appendix}

All of the code used for this report should be included and executed
here. Do not include extraneous code or output not directly relevant to
the results reported in this research report. Use brief headers or
sentences to connect the code to the sections in this report.




\end{document}
